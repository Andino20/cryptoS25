\section{Aufgabe 14}
\textit{Erklären sie das Fuzzy Vault Scheme zur Erzeugung von kryptographischen Schlüsseln
aus biometrischen Messungen. Wann muss dieses Verfahren verwendet werden anstelle
des Fuzzy Commitment Schemes?}\vspace*{1em}\newline
Das Fuzzy Vault Scheme ist ein kryptographisches Verfahren, um Schlüssel mit unscharfen biometrischen Daten 
zu schützen. Im Gegensatz zu klassischer Verschlüsselung, die exakte Eingaben erfordert, 
erlaubt dieses Schema kleine Abweichungen, z.B wenn ein Fingerabdruck-Sensor bei jedem Scan leicht 
unterschiedliche Merkmale erfasst. Das Verfahren ist wie das Fuzzy Commitment Scheme (FCS) in zwei Stufen unterteilt,
der Registrierungs (Enrollment) und der Verifikation. Wie beim Fuzzy Commitment Scheme (FCS) wird während der Registrierung 
ein geheimer Schlüssel generiert und anschließend mit einem fehlerkorrigierenden Code (z. B. Reed-Solomon oder BCH-Code) 
kodiert, um Toleranz gegenüber leichten Abweichungen in den biometrischen Daten zu ermöglichen. Der Code wird als
Polynom kodiert. Beispielsweise würde der Code $7283$ zu einem Polynom mit Grad $3$ werden:
\[
    7283 \longrightarrow p(x)=7x^3 + 2x^2 + 8x^1 + 3
\]
Die biometrischen Merkmale (z.B. Minutien bei Fingerabdrücke) werden als eine Menge $\{x_1,x_2,\dots\}$ von Messpunkten erfasst. Da es
sich um eine Menge handelt, ist die Reihenfolge in der die Daten erfasst werden egal.
Jeder Messpunkt wird nun in das geheime Polynom eingesetzt, um Messpunkt-Ergebnis-Paare zu erhalten.
\[
    \{(x_1, p(x_1)),\ (x_2, p(x_2)),\ \dots\}
\]
Jedes Paar beschreibt einen Punkt auf der Kurve des Polynoms. Zuletzt werden die Punkte um "`falsche"' Punkte (Chaff-Points)
ergänzt. Dadurch kann ein Angreifer das Polynom nicht rekonstruieren, da ihm nicht bekannt ist, welche Punkte 
echte Messpunkte waren. Die Kombination aus den Polynompunkten und den Chaff-Points (auch \textit{Vault} $V$ genannt) wird
gespeichert.
\[
    V = \{(x_1, p(x_1)),\ (x_2, p(x_2)),\ \dots\}\cup \text{Chaff-Points}
\]
Bei der Verifikation werden wieder biometrische Messpunkte erfasst. Diese werden mit $V$ verglichen,
um die Chaff-Points herauszufiltern. Sofern die Daten nah genug an den ursprünglichen sind, kann 
das Polynom rekonstruiert werden. Aus dem Polynom leitet sich dann der Schlüssel ab.
\subsection{FCS vs. FVS}
Da das Fuzzy Vault Scheme mit Messpunkt-Mengen arbeitet, ist die Reihenfolge in der die Punkte gemessen werden
egal. Außerdem ist das FVS unempfindlich gegen das Hinzufügen oder Löschen von Datenpunkten (hohe Rauschtoleranz). 
Daher eignet es sich gut für die Verifikation über Fingermerkmale oder Gesichtserkennung. Chaff-Points bieten außerdem einen starken Schutz gegen
Brute-Force-Angriffe. FVS wird also vorallem dann eingesetzt, wenn ein hohes Sicherheitsmaß notwendig ist. 
Das Fuzzy Commitment Scheme ist besonders geeignet, wenn die Messdaten als Bitstring vorliegen, wie es bei Iris-Codes 
oder binär extrahierten Fingerabdruck-Merkmalen der Fall ist. Das FCS ist außerdem simpler und effizient,
wenn einfache fehlerkorrigierende Codes ausreichen.