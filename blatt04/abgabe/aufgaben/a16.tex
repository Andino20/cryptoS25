\section{Aufgabe 16}
\textit{Implementieren sie die in der VO besprochene short Key XOR Verschlüsselung (Text
wird über ASCII-Nummern binär dargestellt und mit ensprechendem “binärem” Text
Key XOR verschlüsselt, variable Key-länge für Experimente erforderlich). Bestimmen 
sie mit der in der VO besprochenen “Counting Coincidences” Methode die 
Länge des jeweils verwendeten Keys.}\vspace*{1em}\newline
Zuerst muss ein zufällig Schlüssel mit festlegbarer Länger erzeugt werden. Dafür wurde die
Funktion \verb|generate_key| erstellt, die eine gewünschte Schlüssellänge erhält und einen entsprechenden 
Key erzeugt.
\begin{verbatim}
fn generate_key(length: usize) -> Vec<u8> {
    rng().random_iter().take(length).collect::<Vec<u8>>()
}
\end{verbatim}
Der Plaintext kann dann über einen Aufruf der Funktion \verb|short_key_xor| zusammen mit dem Key
verschlüsselt werden. Falls die Länge des Plaintexts kein Vielfaches der Schlüssellänge ist, wird der String
mit dem Zeichen $X$ aufgepolstert. Die Füllzeichen werden während der Iteration über die Zeichen des Plaintexts
angehängt. Die Bytes des Schlüssel wiederholen sich, indem die Position des Zeichens im Plaintext modulo der Schlüssellänge
gerechnet wird. Für die Verschlüsselung selbst werden die Zeichen des Strings in ASCII umgewandelt und mit dem Key geXORt.
\begin{verbatim}
fn short_key_xor(text: &str, key: &[u8]) -> Vec<u8> {
    let padding = key.len() - text.len() % key.len();
    text.chars()
        .chain(std::iter::repeat_n('X', padding))
        .enumerate()
        .map(|(i, c)| c as u8 ^ key[i % key.len()])
        .collect()
}
\end{verbatim}
Um die Länge des Schlüssel zu bestimmen wird der Ciphertext mit jeder möglichen verschobenen Variante von sich selbst
verglichen. In jeder Iteration werden die übereinstimmenden Bytes, für den entsprechenden \verb|offset|, gezählt. 
Ist die relative Häufigkeit der gleichen Bytes $>6.65\%$, so ist \verb|offset| die Schlüssellänge. Wird keine
passende Verschiebung gefunden, wir angenommen, dass der Schlüssel gleich lang wie der originale Text war (Long-Key XOR).
\begin{verbatim}
fn counting_coincidences(text: &[u8]) -> usize {
    for offset in 1..text.len() {
        let same_bytes = text
            .iter()
            .zip(text.iter().chain(text).skip(offset))
            .filter(|&(a, b)| a == b)
            .count() as f32
            / text.len() as f32;
        if same_bytes > 0.06 {
            return offset;
        }
    }
    text.len()
}
\end{verbatim}
Da sich der IOC bei der korrekten Verschiebung $6.65\%$ annähert, wurde hier ein Schwellwert von $>6\%$ festgelegt. Dadurch
werden die korrekten Schlüssellänge auch bei kürzeren Texten identifiziert. Der Algorithmus stoppt sobald ein Wert gefunden wird,
der die Schwelle überschreitet.
\subsection{Ergebnisse}
\textbf{Plaintext}: \textit{Methode von Kasiski: Identische sich wiederholdende Teile des Plaintexts ge-XORed mit dem gleichen Teil des Keys ergeben sich
wiederholende identische Ciphertext Teile. Die Groesse des Shifts
zwischen dem Beginn solcher sich wiederholenden Teile im
Ciphertext sollte daher ein Vielfaches der Key-Laenge sein. Die
Analyse der gemeinsamen Faktoren dieser Shifts identifiziert den
haeufigsten Faktor der dann der Key-laenge entspricht.} (siehe Tabelle \ref{tab:CC_result_1})\vspace*{1em}\newline
Die verwendet Schlüssellänge steht über den Tabellenwerten, die korrekte Verschiebung ist fett markiert. Der letzte Tabelleneintrag
ist das Ergebnisse des Algorithmus, die geschätzte Schlüssellänge.
\begin{table}[h]
    \begin{subtable}[t]{0.3\textwidth}
        \centering
        \begin{tabular}{c|c}
            \multicolumn{2}{l}{\textbf{Keylength: 8}}\\\hline
            SHIFT & IOC\\\hline
            01&      1.14\%\\
            02&      0.45\%\\
            03&      0.45\%\\
            04&      0.00\%\\
            05&      0.45\%\\
            06&      0.00\%\\
            07&      0.23\%\\
            \textbf{08}&      7.95\%
        \end{tabular}
    \end{subtable}
    \begin{subtable}[h]{0.3\textwidth}
        \centering
        \begin{tabular}{c|c}
            \multicolumn{2}{l}{\textbf{Keylength: 13}}\\\hline
            SHIFT & IOC\\\hline
            01 &      0.00\%\\
            02 &      0.23\%\\
            03 &      0.00\%\\
            04 &      0.45\%\\
            05 &      0.23\%\\
            06 &      0.00\%\\
            07 &      0.23\%\\
            08 &      0.23\%\\
            09 &      0.00\%\\
            10 &      0.00\%\\
            11 &      0.68\%\\
            12 &      0.23\%\\
            \textbf{13} &      7.69\%
        \end{tabular}
    \end{subtable}
    \begin{subtable}[h]{0.3\textwidth}
        \centering
        \begin{tabular}{c|c}
            \multicolumn{2}{l}{\textbf{Keylength: 19}}\\\hline
            SHIFT & IOC\\\hline
            01&      0.46\%\\
            02&      0.00\%\\
            03&      0.23\%\\
            04&      0.69\%\\
            05&      0.23\%\\
            06&      0.23\%\\
            07&      0.23\%\\
            08&      0.23\%\\
            09&      0.00\%\\
            10&      0.23\%\\
            11&      0.46\%\\
            12&      0.00\%\\
            13&      0.23\%\\
            14&      0.46\%\\
            15&      0.23\%\\
            16&      0.00\%\\
            17&      0.23\%\\
            18&      0.00\%\\
            \textbf{19}&      6.64\%
        \end{tabular}
    \end{subtable}
    \caption{IOC Werte für unterschiedliche Keylängen, bei unterschiedliche Shifts}
    \label{tab:CC_result_1}
\end{table}\newline
Verwendet man einen kürzeren Plaintext, hat das Verfahren nicht bei jeder Schlüssellänge funktioniert.
Es sind nicht genug Werte vorhanden, um den IOC ausreichend an den Schwellwert anzunähern. Manchmal wurde
nur ein Vielfaches der Schlüssellänge erkannt, oder kein Wert hat den Schwellwert erreicht.\vspace*{1em}\newline
\textbf{Plaintext}: \textit{Wird nicht ein kurzer repetitiver Schluessel verwendet sondern einer der
die gleiche Laenge aufweist wie der Plaintext, spricht man von OTP
Verschluesselung"} (siehe Tabelle \ref{tab:CC_result_2})
\begin{table}[h]
    \begin{subtable}[t]{0.25\textwidth}
        \centering
        \begin{tabular}{c|c}
            \multicolumn{2}{l}{\textbf{Keylength: 8}}\\\hline
            SHIFT & IOC\\\hline
            01&      0.62\%\\
            02&      0.00\%\\
            03&      0.00\%\\
            04&      0.62\%\\
            05&      0.00\%\\
            06&      0.62\%\\
            07&      0.00\%\\
            \textbf{08}&      8.12\%
        \end{tabular}
        \caption{Schlüssellänge konnte richtig identifiziert werden.}
    \end{subtable}
    \hfill
    \begin{subtable}[h]{0.25\textwidth}
        \centering
        \begin{tabular}{c|c}
            \multicolumn{2}{l}{\textbf{Keylength: 12}}\\\hline
            SHIFT & IOC\\\hline
            01&      0.00\%\\
            02&      0.60\%\\
            03&      1.79\%\\
            04&      0.60\%\\
            05&      0.60\%\\
            06&      0.60\%\\
            07&      1.19\%\\
            08&      1.19\%\\
            09&      1.19\%\\
            10&      0.00\%\\
            11&      0.00\%\\
            \textbf{12}&      5.95\%\\
            13&      0.60\%\\
            14&      0.00\%\\
            15&      0.00\%\\
            16&      1.19\%\\
            17&      0.60\%\\
            18&      0.60\%\\
            19&      0.00\%\\
            20&      0.60\%\\
            21&      0.60\%\\
            22&      0.60\%\\
            23&      0.60\%\\
            24&      7.74\%
        \end{tabular}
        \subcaption{Schlüssellänge wurde als ein Vielfaches der richtigen Länge identifiziert.}
    \end{subtable}
    \hfill
    \begin{subtable}[h]{0.25\textwidth}
        \centering
        \begin{tabular}{c|c}
            \multicolumn{2}{l}{\textbf{Keylength: 19}}\\\hline
            SHIFT & IOC\\\hline
            01&      0.58\%\\
            02&      0.58\%\\
            03&      0.00\%\\
            04&      0.00\%\\
            05&      0.58\%\\
            06&      0.58\%\\
            07&      0.00\%\\
            08&      0.00\%\\
            09&      1.17\%\\
            10&      1.17\%\\
            11&      0.00\%\\
            12&      0.00\%\\
            13&      0.00\%\\
            14&      0.00\%\\
            15&      0.00\%\\
            16&      0.00\%\\
            17&      0.58\%\\
            18&      0.00\%\\
            \textbf{19}&      4.68\%\\
            20&      0.58\%\\
            $\vdots$ & $\vdots$\\
            168&     0.00\%\\
            169&     0.58\%\\
            170&     0.58\%
        \end{tabular}
        \subcaption{Schlüssellänge konnte nicht gefunden werden}
    \end{subtable}
    \caption{Zweites Experiment mit kürzerem Text. Nicht jede Schlüssellänge wird richtig erkannt.}
    \label{tab:CC_result_2}
\end{table}
\subsection{Fazit}
Sofern ausreichend Text vorhanden ist, bietet die Counting Coincidences Methode einen guten Weg um die Schlüssellänge
zu bestimmen. Falls der Text zu kurz ist, enstehen falsche Ergebnisse, die entweder ein Vielfaches der richtigen Länge sind,
oder den Schlüssel gleich lang wie den Text schätzen.