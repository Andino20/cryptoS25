\section{Aufgabe 15}
\textit{Implementieren sie als einfaches Bildverschlüsselungsverfahren eine Permutation der
Zeilen eines Bildes. Dazu soll als Parameter (z.B. 1, 2, 4, ...) eine Anzahl von
horizontalen Bildblöcken definiert werden können, innerhalb derer jeweils die Permutationen angewendet werden (also Permutationen auf das ganze Bild für 1, Permutationen innhalb der oberen und unteren Bildhälfte für 2, u.s.w.). Versuchen
sie anschliessend, unter Ausnutzung der Tatsache dass ähnliche Bildzeilen meist
nebeneinander liegen, eine Ciphertext only Attacke gegen das verschlüsselte Bild
(für verschiedene Parameterwerte und Bilder).}\vspace*{1em}\newline
Die Implementierung der Ver- und Entschlüsselung wurde vollständig in der Programmiersprache
\textbf{Rust} geschrieben. Zur Bildmanipulation wird das \href{https://docs.rs/image/latest/image/}{image} Crate verwendet. 
\subsection{Verschlüsselungsverfahren}
Um ein Bild zu verschlüsseln, wird das Bild aus dem Speicher geladen und durch \verb|permutation_cipher| mit der
gewünschten Blockanzahl verschlüsselt. Für
dieses Experiment werden $1,2,4,8,16,32$ Blöcke mit unterschiedlichen Bildern getestet. 
\begin{verbatim}
let original = image::open(&path).expect("Failed to open image");

for blocks in [1, 2, 4, 8, 16, 32] {
    let cipher_image = permutation_cipher(&original, blocks);
    cipher_image
        .save(format!("./img/out/{:02}_{}", blocks, filename))
        .expect("Failed to save image");
}
\end{verbatim}
Normalerweiße würde man das Verschlüsselungsverfahren deterministisch anhand eines
Schlüssel durchführen. Für den Zweck dieses Experiments soll allerdings 
kein Schlüssel ermittelt werden, also können die Zeilen innerhalb eines Blocks
einfach gemischt werden. Dafür werden einfach zwei Zeilen eines Blocks zufällig ausgewählt und
getauscht. Dieser Vorgang wird dann 1000 Mal wiederholt, um einen gemischten Block zu erhalten.
\begin{verbatim}
fn permutation_cipher(original: &image::DynamicImage, blocks: u32) 
    -> image::DynamicImage {
    let mut img = original.to_rgba8();
    let height = img.height();
    let rows_per_block = height / blocks;

    for block in 0..blocks {
    
\end{verbatim}
\subsection{Ciphtext-only Attacke}
\subsection{Ergebnisse}