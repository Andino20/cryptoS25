\documentclass[a4paper,12pt]{article}

\usepackage[utf8]{inputenc}
\usepackage[T1]{fontenc}
\usepackage[ngerman]{babel}
\usepackage{graphicx}
\usepackage{amsmath,amsthm, amssymb}
\usepackage{mathtools}
\usepackage{xcolor}
\usepackage{minted}
\usepackage{lmodern}
\usepackage[a4paper,margin=2.5cm]{geometry}
\usepackage{fancyhdr}
\usepackage{setspace}
\usepackage{caption}
\usepackage{subcaption}
\usepackage{nicefrac}
\usepackage{svg}
\usepackage{hyperref}

% Minted customization to remove background
\usemintedstyle{vs} % Visual Studio-like
\setminted{
    bgcolor=white, % ensures white background
    frame=lines,
    framesep=2mm,
    baselinestretch=1.2,
    fontsize=\footnotesize,
    linenos
}

\theoremstyle{definition}
\newtheorem{example}{Example}
\newtheorem{korollar}{Korollar}

\newcommand{\bigO}{\mathcal{O}}

\pagestyle{fancy}
\fancyhf{}
\lhead{PS Einführung in die Kryptographie}
\rhead{Andreas Schlager}
\cfoot{\thepage}

\title{Aufgabenblatt 09\\\large PS Einführung in die Kryptographie}
\author{Andreas Schlager}
\date{\today}

\onehalfspacing

% Theorem-Umgebung für Aufgaben
\newtheorem{aufgabe}{Aufgabe}

% ---- Makro für neue Aufgabe ----
\newcommand{\newaufgabe}[1]{
  \newpage
  \begin{aufgabe}
  #1
  \end{aufgabe}
  \addcontentsline{toc}{section}{Aufgabe \theaufgabe}
}

\begin{document}

\maketitle
\tableofcontents
\newpage

% Startnummer anpassen (z.B. erste Aufgabe hat Nummer 20, dann counter auf 19 setzen)
\setcounter{aufgabe}{30}

% ---- Aufgaben ----
\newaufgabe{
    Recherchieren sie ein weiteres Kriterium zur Bestimmung einer Primitivwurzel (zu
dem auf Slide 150 der VO) und implementieren sie Experimente, in denen sie, für
wachsende Modulgrösse, den Zeitbedarf beider Kriterien bestimmen. Hinweis: z.B.
in Mathematica sind diverse hilfreiche zahlentheoretische Funktionen - wie z.B. 
Faktorisierung - implementiert.
}
\newaufgabe{
    Beweisen sie die Korrektheit der RSA Ver- und Entschlüsselungsformel für 
    $(m_i, n) = 1$ und $(m_i, n) = 1$.
}
\newaufgabe{Warum ist RSA in der bisherigen Beschreibung (sog. Textbook RSA) 
nicht IND-CPA? Wie wird mit RSA typischerweise diese Sicherheitsstufe erreicht?}

\end{document}
