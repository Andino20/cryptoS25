\newaufgabe{
    Beweisen sie die Korrektheit der RSA Ver- und Entschlüsselungsformel für 
    $(m_i, n) = 1$ und $(m_i, n) = 1$.
}
Die Formeln sind korrekt, wenn in beiden Fällen eine Entschlüsselung nach der Verschlüsselung wieder zum
ursprünglichen Klartext führt.
Seien $p$ und $q$ zwei zufällige, große Primzahlen und $n=pq$. Eine Nachricht $m < n$ 
kann durch
\begin{equation}
    \label{eq:encryption}
    c = m^e\mod n
\end{equation}
verschlüsselt werden, um den Ciphertext $c$ zu erhalten. Die Entschlüsselung erfolgt dann über
\begin{equation}
    \label{eq:decryption}
    m = c^d\mod n.
\end{equation}
$e$ und $d$ sind so gewählt, dass $ed\equiv 1 \mod \varphi(n)$.
\paragraph{Fall 1} $(m,n)=1$, d.h. $m$ und $n$ sind teilerfremd. Die eulersche Verallgemeinerung
des kleinen Fermats besagt, wenn $m$ und $n$ teilfremd sind, dann ist
\begin{equation}
    m^{\varphi(n)}\equiv 1 \mod n.
\end{equation}
Wenden wir die Entschlüsselungsformel \ref{eq:decryption} an, so erhalten wir
\begin{gather*}
    m = c^d \mod n = m^{e^d} \mod n = m^{ed}\mod n\\
    m \equiv m^{ed} \mod n
\end{gather*}
wobei $e,d$ so gewählt wurden, dass
\[
    ed\equiv 1 \mod \varphi(n) \Longleftrightarrow ed = 1 + k\varphi(n)
\]
Wir setzen also für $ed$ ein
\begin{align*}
    m^{ed} &\equiv m^{1 + k\varphi(n)}\mod n\\
    &\equiv m\cdot \left(m^{\varphi(n)}\right)^k \mod n\\
    &\equiv m\cdot (1)^k \mod n\\
    &\equiv m \mod n
\end{align*}
\paragraph{Fall 2} $(m,n)\neq 1$, d.h. $m$ und $n$ sind \textbf{nicht} teilerfremd. Somit ist
$m$ ein Vielfaches von $p$ oder von $q$. Wir können o.B.d.A. annehmen, dass $m$ ein Vielfaches von $p$ ist.
Somit gilt \[
    m \equiv 0 \equiv m^{ed} \mod p \Longrightarrow m\equiv m^{ed}\mod p
\]
Da $m$ nur ein Vielfaches von entweder $p$ oder $q$ sein kann und $q$ eine Primzahl ist, sind $q$ und $m$ teilerfremd.
Der kleine Satz von Fermat besagt, dass für eine Primzahl und einer teilerfremden Zahl, die kein Vielfaches ist
\[
    m^{q-1}\equiv 1\mod q
\]
gilt.
\begin{align*}
    m^{ed} &\equiv m^{1 + k\varphi(n)}\mod q\\
    &\equiv m^{1 + k(p-1)(q-1)}\mod q && \mid k' := k(p-1)\\
    &\equiv m^{1 + k'(q-1)}\mod q\\
    &\equiv m\cdot \left(m^{(q-1)}\right)^{k'}\mod q && \mid \text{kleiner Fermat}\\
    &\equiv m\cdot (1)^{k'}\mod q\\
    &\equiv m \mod q
\end{align*}
Wir wissen also, dass 
\[
    m^{ed} \equiv m \mod p\qquad\text{und}\qquad m^{ed} \equiv m \mod q.
\]
Somit ist auch
\[
    m^{eq} \equiv m \mod pq \Longleftrightarrow m^{eq} \equiv m \mod n.
\]