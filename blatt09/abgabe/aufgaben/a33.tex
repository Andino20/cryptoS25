\newaufgabe{Warum ist RSA in der bisherigen Beschreibung (sog. Textbook RSA) 
nicht IND-CPA? Wie wird mit RSA typischerweise diese Sicherheitsstufe erreicht?}
Reguläres (Textbook RSA) ist nicht IND-CPA sicher, weil es deterministisch ist.
Das heißt, der gleiche Plaintext wird immer gleich verschlüsselt. Falls der Angreifer
in der Lage ist, zwei Ciphertexte mit einer Wahrscheinlichkeit $> \nicefrac{1}{2}$ zu unterscheiden,
ist die Verschlüsselung nicht IND-CPA sicher. 

Angenommen der Angreifer ist in der Lage den Klartextraum einzuschränken, weil er 
etwa weiß, dass nur 'ja' oder 'nein' als Nachrichten verschickt werden. Ausgehend von dem Orakel-Modell,
könnte er dem Orakel beide Texte zum Verschlüsseln geben. Das Orakel wählt zufällig
einen der Texte aus, verschlüsselt ihn mit dem öffentlichen RSA-Schlüssel und schickt ihn
an den Angreifer zurück. Der Angreifer kann nun selbst beide Nachrichten mit dem bekannten
öffentlichen Schlüssel verschlüssen und einen einfachen Vergleich der Ciphertexte durchführen,
um den ausgewählten Klartext herauszufinden.

Um RSA IND-CPA sicher zu machen, muss der Plaintext mit zufälligen Bits ausgepolstert werden. Dadurch
wird sich der Ciphertext, den der Angreifer generiert, vom Ciphertext des Orakels unterscheiden.
In der Praxis wird das z.B durch OAEP (Optimal Asymmetric Encryption Padding) realisiert.