\newaufgabe{
    Recherchieren sie ein weiteres Kriterium zur Bestimmung einer Primitivwurzel (zu
dem auf Slide 150 der VO) und implementieren sie Experimente, in denen sie, für
wachsende Modulgrösse, den Zeitbedarf beider Kriterien bestimmen. Hinweis: z.B.
in Mathematica sind diverse hilfreiche zahlentheoretische Funktionen - wie z.B. 
Faktorisierung - implementiert.}
Eine Primitivwurzel $q$ modulo $p$ ist eine Zahl im Restklassenring von $p$,
die durch modulares potenzieren alle Zahlen im Restklassenring erzeugt.
In der Vorlesung wurde ein einfaches Kriterium für die Bestimmung einer
Primitivwurzel vorgestellt. Sei $q < p$ und
\[
    \forall b\quad 1 \leq b \leq p - 1\quad \exists a: q^a \equiv b\mod p,
\]
so ist $q$ eine Primitivwurzel von $p$. Also kann durch einfaches Ausprobieren 
aller Exponenten modulo $p$ festgestellt werden, ob $q$ eine Primitivwurzel
ist. 
\paragraph{Brute-Force Algorithmus}
Es wird angenommen, dass die gegebene Zahl $p$ wirklich eine Primzahl ist. Für die Überprüfung könnte
ein probabilistischen Primzahlentest, wie etwa der Miller-Rabin-Test, verwendet werden.
Für jede Zahl $q\in [2; p - 1]$ und $i\in[1; p-1]$ wird die modulare Exponentiation durchgeführt
und das Ergebnis in einem Set gespeichert. Ist das Ergebnis bereits vorhanden, bzw. kommt doppelt vor,
so kann $q$ keine Primitivwurzel sein, da jeder Exponent zu einer einzigartigen Zahl führen muss.
\begin{minted}{rust}
pub fn primitive_root_brute_force(p: u128) -> Option<u128> {
    let phi = p - 1;
    for q in 2..phi {
        let mut unique_numbers = HashSet::new();
        let unique_numbers = (1..=phi)
            .take_while(|x| unique_numbers.insert(mod_exp(q, *x, p)))
            .count();

        if unique_numbers == phi as usize {
            return Some(q);
        }
    }
    None
}
\end{minted}
\paragraph{Ordnungskriterium}
Ein weiteres Kriterium für eine Primitivwurzel verwendet die multiplikative Ordnung einer
Zahl und die eulersche $\varphi$-Funktion.
\begin{korollar}[Primitivwurzel]
    Sei $q < p$ und $p\in\mathbb{P}$. $q$ ist genau dann eine Primitivwurzel modulo $p$, wenn
    \[
        \operatorname{ord}_p(q) = \varphi(p) = p - 1. 
    \]
\end{korollar}
Die Ordnung $\operatorname{ord_p(q)}$ ist der kleinste Exponent $i\in[1;p-1]$ für
den
\[
    q^i \equiv 1 \mod p
\]
Der Satz von Lagrange besagt, dass die Ordnung eines Elements einer endlichen zyklischen Gruppe die Ordnung
der Gruppe selbst, hier $\varphi(p)$, teilt. Man nutzt nun diese Eigenschaft aus und probiert anstelle aller Exponenten
nur die Teiler von $\varphi(p)$ (alle möglichen Ordnungen der Gruppe). Ist der kleinste Exponent $i$ für den $q^i\equiv 1\mod p$ gilt,
gleich $p-1$, so ist $q$ eine Primitivwurzel.

Diese Vorgehen kann noch weiter verbessert werden, indem nur die Primfaktoren von $\varphi(p)$ berechnet
werden und 
\begin{equation} \label{eq:primefactorcriterium}
    q^{\nicefrac{p-1}{g}} \not\equiv 1 \mod p\qquad\forall \text{ Primfaktoren } g \mid p - 1    
\end{equation}
überprüft wird\footnote{Otto Forster, \href{https://www.mathematik.uni-muenchen.de/~forster/v/zth/inzth_09.pdf}{Einführung in die Zahlentheorie}, [22.05.2025], Korollar 9.5}.
\paragraph{Lagrange-Primfaktoren-Algorithmus}
Zuerst werden die Primfaktoren von $p-1$ bestimmt. Anschließend wird, wie beim Brute-Force Algorithmus, über alle
$q\in[2;p-1]$ iteriert und das Kriterium aus \ref{eq:primefactorcriterium} überprüft. Falls keine Berechnung 1 ergibt,
liefert die Methode als Ergebnis das aktuelle $q$. In den Zeilen 5-8 ist zu sehen, wie die Primfaktoren auf die
modulare Exponentiation gemapped und anschließend mittels \texttt{all} auf die Ungleichheit zu 1 überprüft werden.
\begin{minted}{rust}
pub fn primitive_root_lagrange(p: u128) -> Option<u128> {
    let phi = p - 1;
    let prime_factors = Factorization::run(phi).prime_factor_repr();
    for q in 2..phi {
        let is_primitive_root = prime_factors
            .iter()
            .map(|(factor, _)| mod_exp(q, phi / factor, p))
            .all(|x| x != 1);

        if is_primitive_root {
            return Some(q);
        }
    }
    None
}
\end{minted}
\newpage
\paragraph{Benchmarks} Zum Vergleich der zwei Algorithmen wurde die \texttt{criterion} Benchmarking-Bibliothek verwendet.
Jede Funktion wurde mit einer größer werdenden Folge von Primzahlen aufgerufen und der Zeitbedarf gemessen. Die verwendete
Primzahlen Liste ist überschaubar, da die Laufzeiten, vorallem von der Brute-Force Variante, schnell gestiegen sind.
\[
    \text{primes} = \{223, 1033, 2239, 5179, 7919, 22447, 73907, 111799, 153913\}
\]
\begin{figure}[h]
    \includesvg[inkscapelatex=false, width=\textwidth]{img/time_comparison.svg}
    \caption{Vergleich der Laufzeiten bei ansteigenden Primzahlen}
    \label{fig:runtime_comparison}
\end{figure}
Wie in Abb. \ref{fig:runtime_comparison} zu sehen ist, steigt die Laufzeit der Brute-Force Variante besonders schnell
im Vergleich zum Lagrange-Primfaktoren-Algorithmus (grüne Linie, kaum zu sehen). Der verbesserte Algorithmus
bewegt sich durchwegs im $\mu$s Bereich, wobei der langsame schnell Millisekunden bzw. Sekunden benötigt.
\begin{figure}
    \includesvg[inkscapelatex=false, width=\textwidth]{img/lagrange_violin.svg}
    \caption{Violinen-Plot der Laufzeit der Lagrange-Primfaktoren Variante}
    \label{fig:violin_lagrange}
\end{figure}
\begin{figure}
    \includesvg[inkscapelatex=false, width=\textwidth]{img/brute_force_violin.svg}
    \caption{Violinen-Plot der Laufzeit der Brute-Force Variante}
    \label{fig:violin_brute_force}
\end{figure}
In Abb. \ref{fig:violin_lagrange} ist nochmal deutlich zu sehen, 
dass die Laufzeit bei der Lagrange-Variante nur langsam ansteigt.
In Abb. \ref{fig:violin_brute_force} ist hingegen ein deutlicher Anstieg der Laufzeit zu erkennen.