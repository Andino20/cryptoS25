\newaufgabe{Zur Known-Plaintext Attacke gegen 2 Key EDE TripleDES (Slide 138f): 
Erklären/beweisen
sie die angegebene Angriffskomplexität / Laufzeit der Oorschot, Wiener Attacke und
die Anzahl der im Mittel notwendigen Versuche für a. Hinweis: sehen sie sich die
Originalarbeit dazu an.}
\paragraph{Zeitkomplexität}
Zur Berechnung der asymptotischen Zeitkomplexität des Angriffs, wird für jeden 
Teilschritt die Komplexität angegeben.
\begin{enumerate}
    \item Gegeben eine Tabelle mit Plaintext- und Ciphertext Paaren
          (Tabelle 1, sortiert nach Plaintexten).
          \begin{enumerate}
            \item Sortieren benötigt bei $p$ Paaren $\bigO (p\log p)$.
          \end{enumerate}
    \item Erraten/wählen Sie einen fixen ersten Zwischenwert $a = E_{K_1}(P)$.
          \begin{enumerate}
            \item Ein fixes $a$ wählen geht in $\bigO (1)$.
          \end{enumerate}
    \item Bestimmen sie durch die Berechnung von $P = D_{K_1} (a)$ den
          Plaintext und suchen den dazugehörenden Ciphertext $C$ in Tabelle
          1, dies für alle $K_1$.
          \begin{enumerate}
            \item Pro $K_1$:
            \begin{enumerate}
                \item Berechnung von $P = D_{K_1} (a)$ in $\bigO (1)$.
                \item In der Tabelle suchen $\bigO (\log p)$.
            \end{enumerate}
            \item Insgesamt gibt es für $K \in \{0, 1\}^n$ $2^n$ Durchläufe. 
          \end{enumerate}
          Somit ist die Gesamtkomplexität für diesen Schritt $\bigO(2^n\log p)$.
    \item Tabulieren Sie für alle $K_1$ (für das fixe $a$) den zweiten
          Zwischenwert, $b = D_{K_1} (C)$ (Tabelle 2).\vspace*{1em}\\
          Entspricht wieder einem gesamten Durchlauf der Schlüssel ($2^n$). Bei jedem wird
          $b$ in $\bigO(1)$ berechnet. Die Komplexität beträgt daher $\bigO(2^n)$.
    \item Suchen Sie in der Tabelle 2 für alle möglichen $K_2$ Elemente mit
          passendem zweiten Zwischenwert $b = D_{K_2} (a)$.
          \begin{enumerate}
            \item Pro $K_2$:
            \begin{enumerate}
                \item Berechne $b = D_{K_2}(a)$ in $\bigO(1)$.
                \item Prüfe $b$ in Tabelle 2 (angenommen Hashtabelle) in $\bigO(1)$.
            \end{enumerate}
            \item Insgesamt gibt es wieder $2^n$ $K_2$, also auch $2^n$ Durchläufe.
          \end{enumerate}
          Gesamtkomplexität ist $\bigO(2^n)$.
    \item Bei Übereinstimmung ist ein mögliches Schlüsselpaar gefunden
          (mit weiteren Plaintext / Ciphertextpaaren verifizieren). Ansonsten
          wird ein neues $a$ gewählt.\vspace*{1em}\\
          Verifikation durch $p$ bekannte Paare in $\bigO(p)$.
\end{enumerate}
Durch die Schritte 3, 4 \& 5 ergeben sich folgende Laufzeiten
\[
    \bigO(2^n\log p), \bigO(2^n), \bigO(2^n).
\]
Im schlechtesten Fall müssen alle Zwischenwert für $a\in\{0,1\}^m$ überprüft werden.
\[
    \bigO(2^n\cdot 2^m) = \bigO(2^{n + m})
\]
\paragraph{Wahrscheinlichkeit} Es existieren $2^m$ mögliche Werte für $a$. In 
der Datenbank befinden sich $p$ Einträge. Unter der Annahme, dass jeder
Zwischenwert der Klartextpaare gleichverteilt ist, ist die Wahrscheinlichkeit einen
Treffer zu landen 
\[
    P(\text{Treffer})=\frac{p}{2^m}.
\]
Nimmt man dieses Wahrscheinlichkeit als die Wahrscheinlichkeit für ein Bernoulli-Experiment an,
gibt uns die geometrische Verteilung den Erwartungswert für die Anzahl $N$ an unterschiedlichen
Werten für $a$.
\[
    \mathbb{E}[N] = \frac{1}{P(\text{Treffer})} = \frac{2^m}{p}
\]
Im Schnitt müssen wir also nicht alle $2^m$ Werte für $a$ probieren (wie im worst-case),
sondern nur $\mathbb{E}[N]$ viele. Die erwartete Laufzeitkomplexität ist
\[
    \bigO\left(2^n\cdot \frac{2^m}{p}\right) = \bigO\left(\frac{2^{n + m}}{p}\right) 
\]
\paragraph{Fazit}
Je größer die Datenbank an Plaintext-Ciphertext-Paaren ist, desto schneller geht der Angriff.
