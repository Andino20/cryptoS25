\newaufgabe{Beweisen Sie: Ist $n=p\cdot g$ mit $p,g\in \mathbb{P}$: $\phi(n) = (p - 1)(g - 1)$.}
\begin{proof}
    $\phi(n) = (p - 1)(g - 1)$.

    $n$ ist eine zusammengesetzte Zahl, deren Primfaktorenzerlegung 
    genau $p \cdot g$ ist. Somit sind alle Vielfachen der Primzahlen, die
    nicht größer als $n$ sind, \textbf{nicht} teilerfremd zu $n$ (der gemeinsame Teiler
    sind die entsprechenden Primzahlen). 
    \begin{align*}
        \vert 
        \{
            k\cdot p \mid 1 \leq k \leq g 
        \} \vert = \vert \{
            p, 2p, 3p, \dots, g\cdot p
        \}\vert &= g\\
        \vert
        \{
            k\cdot g \mid 1 \leq k \leq p 
        \} \vert = \vert \{
            g, 2g, 3g, \dots, p\cdot g
        \}\vert &= p
    \end{align*}
    Ausgehend von alle in Frage kommenden Zahlen $p\cdot g$, ziehen wir
    jeweils obigen Anzahl ab und achten darauf, die Schnittmenge ($n$) nur einmal
    abzuziehen.
    \begin{align*}
            \phi(n) &= p\cdot g - 1 - p + 1 - g + 1\\
            &= p\cdot g - p - g + 1\\
            &= (p - 1) (g - 1)
    \end{align*}
\end{proof}