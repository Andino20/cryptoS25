\section{Aufgabe 19}
\textit{Leiten Sie die kombinatorischen Werte des Geburtstags-Paradoxons her.}
\paragraph{1. } Wie viele Menschen müssen in einem Raum sein, damit die Wahrscheinlichkeit
größer $\nicefrac{1}{2}$, dass eine oder mehr Personen an einem vorgegebene
Tag Geburtstags haben?\vspace*{1em}\newline
Sei $k$ die Anzahl der Personen in einem Raum. Wir betrachten das Gegenereignis, dass von den $k$
Menschen keiner an dem vorgegebenen Tag Geburtstag hat. Das heißt an einem der anderen 364 Tage.
\begin{gather*}    
    \text{\# günstige Fälle} = 364^k\\
    \text{\# mögliche Fälle} = 365^k
\end{gather*}
Sei $X$ eine Zufallszahl, die beschreibt wie viele von $k$ möglichen Menschen an einem vorgegebenen
Tag Geburtstags haben.
\begin{align*}    
    P(X\ge 1) &= 1 - P(X=0) = 1 - \left(\frac{364}{365}\right)^k\\
    0.5 &= 1 - \left(\frac{364}{365}\right)^k\\
    \ln0.5 &= \ln\left(\frac{364}{365}\right)^k= k\cdot \ln\left(\frac{364}{365}\right)\\
    k &= \frac{\ln 0.5}{\ln \frac{364}{365}} = 252.652
\end{align*}
Also ab 253 Personen wäre die Wahrscheinlichkeit, dass zumindest eine Person an einem
vorgegebenen Tag Geburtstag hat größer \nicefrac{1}{2}.
\paragraph{2. } Wie viele Menschen müssen in einem Raum sein,
damit die Wahrscheinlichkeit größer \nicefrac{1}{2} ist, dass zwei
Personen am gleichen Tag Geburtstag haben?\vspace*{1em}\newline
Wir betrachten das Gegenereignis, dass keine zwei der $k$ Personen am gleichen Tag 
Geburtstag haben. Die Anzahl der günstigen Fälle ergibt sich wie folgt:
\[
    \text{\# günstige Fälle} = 365\cdot 364\cdot 363\cdots (365 - k + 1) = \prod_{i = 0}^{k - 1} (365 - i).
\]
Die Gesamtanzahl der möglichen Kombination ist
\[
    \text{\# mögliche Fälle} = 365^k.
\]
Dadurch erhalten wir die Wahrscheinlichkeit, dass alle Personen an unterschiedlichen Tagen
Geburtstag haben.
\[
    P(\text{alle unterschiedlich}) = \frac{\prod_{i = 0}^{k - 1} (365 - i)}{365^k} = \prod_{i = 0}^{k - 1} \frac{365 - i}{365}
\]
Die gesuchte Wahrscheinlichkeit, dass bei $k$ Personen zwei oder mehr am gleichen Tag haben ist dann
\[
    P(\text{zwei oder mehr gleiche}) = 1 - P(\text{alle unterschiedlich}).
\]
Durch numerische Auswertung für verschiedene $k$ erhält man ein Ergebnis $> 0.5$ bei $k = 23$.