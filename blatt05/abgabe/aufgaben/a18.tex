\section{Aufgabe 18}

\textit{Fortsetzung Aufgabe 11.): Führen sie eine Known Plaintext Attacke gegen den in
Aufgabe 11.) implementierten Cipher durch und ermitteln sie aus den erzeugten
Daten automatisch den für z verwendeten Wert}
\vspace*{1em}\newline
Im Gegensatz zur vorherigen Aufgabe steht ein fester, bekannter Klartext zur Verfügung, welcher in einer „Black-Box“ verschlüsselt wird. Die Verschlüsselung erfolgt intern mit einem zufällig gewählten Schlüssel.
\begin{verbatim}
let plain = "World";
let cipher = black_box(plain);
\end{verbatim}
Die Funktion \texttt{black\_box} nimmt einen Klartext entgegen, generiert intern einen zufälligen Schlüssel im Bereich $[0, 25]$ und verschlüsselt den Text mit der Caesar-Chiffre:
\begin{verbatim}
fn black_box(plain: &str) -> String {
    let key = rng().random_range(0..26);
    caesar_cipher(plain, key) // Implementierung aus Aufgabe 11
}
\end{verbatim}
Ein Beispielversuch mit dieser Methode ergibt folgendes Klartext-Ciphertext-Paar:
\begin{verbatim}
Secret Key:     9
Plain:          World
Cipher:         Fxaum
\end{verbatim}
Da es sich bei der Caesar-Verschlüsselung um eine monoalphabetische Substitution mit fester Verschiebung handelt, genügt es, die Differenz eines Buchstabenpaares (aus Plain- und Ciphertext) zu berechnen, um den verwendeten Schlüssel zu ermitteln.
Seien $P$ der numerische Wert eines Buchstabens aus dem Klartext und $C$ der entsprechende Wert im Geheimtext. Der Schlüssel $S$ ergibt sich dann durch:
\[
S = (C - P) \bmod 26
\]
In der Praxis kann diese Berechnung beispielsweise durch Vergleich des ersten Buchstabenpaares durchgeführt werden:
\begin{verbatim}
let c1 = plain.chars().nth(0).unwrap() as i8;
let c2 = cipher.chars().nth(0).unwrap() as i8;
let key = (c2 - c1).rem_euclid(26);
\end{verbatim}
Wendet man dieses Verfahren auf das obige Beispiel an, so ergibt sich:
\begin{verbatim}
Secret Key:     9
Plain:          World
Cipher:         Fxaum
Calculated Key: 9
\end{verbatim}

\subsection*{Fazit}
Ein Known-Plaintext-Angriff auf eine Caesar-Verschlüsselung lässt sich mit minimalem Aufwand erfolgreich durchführen. 
Bereits ein einziges Zeichenpaar reicht aus, um den verwendeten Schlüssel eindeutig zu rekonstruieren.