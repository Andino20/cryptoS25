\section{Aufgabe 20}
\textit{An welcher Stelle des Protokollangriffs gegen digitale Empfangsbestätigungen
ist tatsächlich die Bedingung }
\[
    V_x = E_x\qquad S_x = D_x
\]
\textit{für den Erfolg notwendig? Warum funktioniert das ohne diese Bedingung nicht?}
\vspace*{1em}\newline
Nachdem Mallory die Nachricht von Alice abgefangen hat besitzt er $E_B(S_A(m))$.
Wenn Mallory nun die Nachricht an Bob schickt und vorgibt der Urheber zu sein,
entschlüsselt und überprüft Bob die Nachricht. Er erstellt
\[
    V_M(D_B(E_B(S_A(m))))
\]
An dieser Stelle "`entfernt"' Bob sich aus der Gleichung, da sich $D_B(E_B(\dots))$
aufheben. Es bleibt $V_M(S_A(m))$. Wird nun der gleiche Algorithmus zur Signatur/Verifikation
und zur Ver- und Entschlüsselung verwendet gilt $V_M = E_M$. 
Somit hätte Bob eigentlich $E_M(S_A(m))$. Anschließend schickt Bob die 
signierte Empfangsbestätigung an Mallory.
\[
    V_M(S_B(E_M(S_A(m)))) = E_M(S_B(E_M(S_A(m))))
\]
Mallory kann ohne Schwierigkeiten $E_M(\dots)$ durch $D_M(E_M(\dots))$ entfernen.
Es bleibt 
\[
    S_B(E_M(S_A(m)))
.\]
Aufgrund der Bedingung $S_x = D_x$ ist eine Signatur gleich einer Entschlüsselung mit
dem Privaten Schlüssel.
\[
    S_B(E_M(S_A(m))) = D_B(E_M(D_A(m)))
\]
Mallory kann nun wie in der VO beschrieben zuerst den Public Key von Bob zur Verschlüsselung,
dann seinen Private Key zur Entschlüsselung und zuletzt den Public Key von Alice verwenden, um
die Nachricht $m$ zu erhalten.
