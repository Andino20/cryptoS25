\documentclass{article}

\usepackage{inputenc}[utf8]
\usepackage[T1]{fontenc}
\usepackage[a4paper]{geometry}
\usepackage{fancyhdr}
\usepackage{url}
\usepackage{hyperref}
\usepackage[ngerman]{babel}
\usepackage{graphicx}

\usepackage{amsmath}

\title{{\Huge Aufgabenblatt 03}\\Einführung in die Kryptographie PS}
\author{Andreas Schlager}


\begin{document}
    \pagestyle{fancy}
    \fancyhead{}
    \fancyhead[L]{Aufgabenblatt 03}
    \fancyhead[R]{Einführung in die Kryptographie}
    \fancyfoot{}
    \fancyfoot[L]{Andreas Schlager}
    \fancyfoot[R]{\thepage}
    \maketitle
    \tableofcontents
    \section{Aufgabe 10}
    \textit{Fortsetzung Aufgabe 9.) Simulieren sie verschiedene Arten von biometrischer Varianz, 
    die sich als gleichverteilte Bitfehler steigender Anzahl oder Bursts (gehäufte
    Fehler an einer oder mehreren Stellen) manifestieren. Wenden sie verschiedene
    Hamming-Codes zur Fehlerkorrektur an. Dokumentieren sie die Auswirkung von
    verschiedenen Fehlerarten (Quantität, Qualität) auf die Möglichkeit, den Schlüssel
    tatsächlich korrekt zu erzeugen.}\vspace*{1em}\newline
    Hallo

    \section{Aufgabe 11}
    \textit{Implementieren sie den Caesar Cipher (Slide 14) mit z als Variable/Schlüssel für
    Buchstaben-orientierte Textverschlüsselung. Führen sie eine (Ciphertext-only) brute
    Force Attacke gegen einen verschlüsselten Text aus (unter der Annahme der Wert
    von z wäre nicht bekannt) und überlegen sie sich ein oder mehrere Kriterien um den
    tatsächlich richtigen Plaintext unter allen erzeugten zu eruieren (und wenden sie das
    alles auf Beispiele an).}\vspace*{1em}\newline
    \newpage
    Welt
    \newpage
    \bibliographystyle{plainnat}
    \bibliography{refs}
\end{document}