\newaufgabe{Fortsetzung Aufgabe 20.): Erklären sie, warum die Verwendung von Hashfunktionen
im Kontext mit digitalen Signaturen die Protokollattacke gegen digitale Empfangsbestätigungen verunmöglicht. Bedenken sie dabei insbesondere, dass in diesem Fall
NachrichtenHash und Nachricht übermittelt werden (die Fragen in Item 2 auf Skriptum S. 93 müssen bearbeitet werden).}
Sei $m$ die Nachricht, die Alice an Bob übermitteln möchte. Alice verschlüsselt 
die Nachricht mit Bobs öffentlichem Schlüssel und sendet ihm $E_B(m)$. 
Zusätzlich erzeugt sie den Hashwert $h(m)$ mittels einer kryptografischen 
One-Way-Hashfunktion, signiert diesen mit ihrem privaten Schlüssel und 
verschlüsselt auch die Signatur mit Bobs öffentlichem Schlüssel:
\[
    \begin{rcases}
        E_B(m)\\
        E_B(S_A(h(m)))
    \end{rcases}
    \rightarrow \text{Bob}
\]
Bob entschlüsselt beide Nachrichtenkomponenten, extrahiert den Klartext $m$ 
sowie die Signatur $S_A(h(m))$, und berechnet anschließend selbst den Hashwert 
$h(m)$. Anschließend überprüft er die Signatur mit Alices öffentlichem Schlüssel. 
Stimmen der berechnete Hash $h(m)$ und der signierte Hash überein, ist die 
Integrität der Nachricht sichergestellt.

Im Anschluss sendet Bob eine digitale Empfangsbestätigung zurück. Diese besteht 
nicht aus einer Signatur der vollständigen Nachricht, sondern aus der Signatur 
des Hashwerts:
\[
    E_A(S_B(h(m)))
\]
Eine Protokollattacke, wie sie in Aufgabe 20 beschrieben wurde, ist in diesem 
Szenario nicht mehr möglich. Grund hierfür sind die 
Eigenschaften der verwendeten Hashfunktion:
\begin{itemize}
    \item \textbf{Kollisionsresistenz:} Es ist praktisch nicht möglich, zwei verschiedene Nachrichten $m \neq m'$ zu finden, sodass $h(m) = h(m')$ gilt.
    \item \textbf{Pre-Image-Resistenz:} Aus einem gegebenen Hashwert $h(m)$ kann keine gültige Nachricht $m$ rekonstruiert werden.
\end{itemize}
Selbst wenn Mallory die Nachrichtenkomponenten vom 
ursprünglichen Protokollangriff kennt und in der Lage wäre, die Signaturen und 
Verschlüsselungen zu trennen, fehlt ihm dennoch die Möglichkeit, aus dem Hashwert 
$h(m)$ die Originalnachricht $m$ zu rekonstruieren oder eine alternative Nachricht 
$m'$ mit demselben Hash zu erzeugen.
Somit verhindert die Einbindung von One-Way-Hashfunktionen diese Art der 
Attacken gegen digitale Empfangsbestätigungen.