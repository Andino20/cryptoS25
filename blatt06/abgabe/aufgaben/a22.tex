\newaufgabe{HÜ10 auf S. 86 der VO-Slides.}
\paragraph{Fall 1} Sei $M$ eine beliebige, aber fixe Nachricht zu der eine
Nachricht $M'$ gefunden werden soll, sodass $h(M) = h(M')$. 
Da die Hashfunktion einen $m$-Bit langen Output produziert, gibt es $2^m$ mögliche 
Hashwerte. Die Wahrscheinlichkeit eine entsprechenden Nachricht $M'$ zu generieren ist
\[
P(h(M) = h(M')) = \frac{1}{2^m}.
\]
Das zufällige erzeugen einer Nachricht von Nachricht $M_1,M_2,\dots$
bis der Hashwert übereinstimmt, entspricht einer geometrischen Verteilung.
\[
    \mathbb{E}[N]=\frac{1}{p}=\frac{1}{\frac{1}{2^m}}=2^m
\]
\paragraph{Fall 2} Man berechnet $M_1,M_2,\dots,M_n$ und will wissen wie viele Nachrichten 
man generieren muss, damit es zumindest eine Übereinstimmung der Hashwerte gibt.
Dafür betrachtet man die Gegenwahrscheinlichkeit, dass alle Hashwerte unterschiedlich sind.
\begin{align*}
    \text{\# günstige Fälle} &= 2^m \cdot (2^m - 1) \cdots (2^m - (n - 1))\\
    \text{\# mögliche Fälle} &= (2^m)^n\\
    P(\text{alle unterschiedlich}) &= \frac{2^m}{2^m}\cdot \frac{2^m - 1}{2^m} \cdots \frac{2^m - (n-1)}{2^m}
    =\prod_{k = 0}^{n-1} \left( 1 - \frac{k}{2^m}\right)
\end{align*}
Wir schreiben das Produkt durch den Logarithmus in eine Summe um.
\begin{align*}
    P(\text{alle unterschiedlich}) &= \prod_{k = 0}^{n-1} \left( 1 - \frac{k}{2^m}\right)\\
    \ln P(\dots) &= \sum_{k=0}^{n-1}\ln\left(1-\frac{k}{2^m}\right)
\end{align*}
Da $k \ll 2^m$ kann für $\ln(1-x)$ die Annäherung $\ln(1-x) \approx -x$ getroffen werden.
\[
    \sum_{k=0}^{n-1}\ln\left(1-\frac{k}{2^m}\right) = -\sum_{k=0}^{n-1}\frac{k}{2^m}
    =-\frac{1}{2^m}\sum_{k=0}^{n-1}k=-\frac{1}{2^m}\cdot\frac{n(n-1)}{2}
\]
Für die Gegenwahrscheinlichkeit bedeutet das
\begin{align*}
    \ln P(\text{alle unterschiedlich}) &\approx -\frac{1}{2^m}\cdot\frac{n(n-1)}{2}\\
    P(\text{alle unterschiedlich}) &\approx \exp \left(-\frac{1}{2^m}\cdot\frac{n(n-1)}{2}\right)
\end{align*}
Wir suchen wieder ein $n$ für das $P(\text{alle unterschiedlich}) \approx \nicefrac{1}{2}$, d.h.
\begin{align*}
    \exp \left(-\frac{1}{2^m}\cdot\frac{n(n-1)}{2}\right) &= \frac{1}{2}\\
    -\frac{1}{2^m}\cdot\frac{n(n-1)}{2} &= \ln \frac{1}{2} = -\ln 2\\
    \frac{n(n-1)}{2^m \cdot 2} &= \ln 2\\
    n(n-1) &= 2^m\cdot 2\cdot \ln 2
\end{align*}
Für große $n$ ist $n(n-1)\approx n^2$
\begin{align*}
    n^2 &\approx 2^m \cdot 2\cdot \ln 2\\
    n &\approx \sqrt{2^m \cdot 2\cdot \ln 2} = \sqrt{2^m} \cdot \sqrt{2\ln 2}\\
    n &\approx 2^{\nicefrac{m}{2}}\cdot\sqrt{2\cdot\ln 2} = 2^{\nicefrac{m}{2}}\cdot 1.1774\dots\\
    n &\approx 2^{\nicefrac{m}{2}}
\end{align*}
